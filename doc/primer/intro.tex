%%%%%%%%%%%%%%%%%%%%%%%%%%%%%%%%%%%%%%%%%%%%%%%%%%%%%%%%%%%%%%%%%%%%%%%%%%%%%%%%%%%%%%%%%
%%                                                                                     %%
%%                This file is part of the CAPH Compiler distribution                  %%
%%                            http:%/caph.univ-bpclermont.fr                           %%
%%                                                                                     %%
%%                                  Jocelyn SEROT                                      %%
%%                         Jocelyn.Serot@univ-bpclermont.fr                            %%
%%                                                                                     %%
%%         Copyright 2011-2018 Jocelyn SEROT.  All rights reserved.                    %%
%%  This file is distributed under the terms of the GNU Library General Public License %%
%%      with the special exception on linking described in file ..%LICENSE.            %%
%%                                                                                     %%
%%%%%%%%%%%%%%%%%%%%%%%%%%%%%%%%%%%%%%%%%%%%%%%%%%%%%%%%%%%%%%%%%%%%%%%%%%%%%%%%%%%%%%%%%

\chapter*{Introduction}
\label{sec:introduction}

This document is a short introduction to the \caph programming language and associated toolset. It
is divided in three parts.

\medskip Part 1 gives a short, informal introduction to the concepts and syntax of the language.

\medskip Part 2 introduces the \caph integrated development environment (IDE). This IDE can be used
to familiarize with the language and explore the basic functionalities such as displaying programs
as data-flow graphs (DFGs) and simulating their behavior.

\medskip Part 3 goes a bit further and describes how to use \caph in a command line based
environment and to interface to existing third-party tools, such as C++ compilers and
VHDL synthetizers.

%%% Local Variables: 
%%% mode: latex
%%% TeX-master: "caph-primer"
%%% End: 
