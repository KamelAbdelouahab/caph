%%%%%%%%%%%%%%%%%%%%%%%%%%%%%%%%%%%%%%%%%%%%%%%%%%%%%%%%%%%%%%%%%%%%%%%%%%%%%%%%%%%%%%%%%
%%                                                                                     %%
%%                This file is part of the CAPH Compiler distribution                  %%
%%                            http:%/caph.univ-bpclermont.fr                           %%
%%                                                                                     %%
%%                                  Jocelyn SEROT                                      %%
%%                         Jocelyn.Serot@univ-bpclermont.fr                            %%
%%                                                                                     %%
%%         Copyright 2011-2018 Jocelyn SEROT.  All rights reserved.                    %%
%%  This file is distributed under the terms of the GNU Library General Public License %%
%%      with the special exception on linking described in file ..%LICENSE.            %%
%%                                                                                     %%
%%%%%%%%%%%%%%%%%%%%%%%%%%%%%%%%%%%%%%%%%%%%%%%%%%%%%%%%%%%%%%%%%%%%%%%%%%%%%%%%%%%%%%%%%

\chapter{Foreign function interfacing}
\label{cha:foreign-funct-interf}

This chapter describes how to make use of foreign functions in \caph program using a simple example.

Consider the following program, in which the multiplication in the \texttt{scale} actor is performed
by calling a foreign (external) function\footnote{The purpose of the example is only to
  demonstrate how to use the foreign function facilities; the \texttt{scale} can of course be
  written in a more straightforward manner using the builtin multiplication operator, like
  exemplified in Sec.~\ref{sec:actor-examples} for example.} : 

\begin{example}
function mult = extern "mult", "mult", "mult" :
  signed<s> * signed<s> -> signed<s>;

actor scale (k:signed<8>)
  in (a:signed<8>)
  out (c:signed<8>)
rules
| a:v -> c:mult(v,k)

stream i:signed<8> from "sample.txt";
stream o:signed<8> to "result.txt";

net o = scale [2] i;
\end{example}

\medskip
Compiling this program with the \textbf{SystemC} back-end, with :

\begin{center}
  \begin{alltt}
   caph -systemc scale.cph
  \end{alltt}
\end{center}

will produce (among others -- see Chap.~\ref{cha:using-caph-compiler}) a file
\texttt{scale\_act.cpp} containing the implementation of the SystemC module describing
the \texttt{scale} actor and containing a call to a function
\texttt{sc\_int<8> mult(sc\_int<8> x, sc\_int<8> y)}.
This function must be
declared (resp. defined) in a file named \texttt{extfns.h} (resp. \texttt{extfns.cpp}). These two
files must be accessible when compiling the SystemC executable. Here's a possible (and obvious)
contents for these files :

\begin{lstlisting}[language=C,title=File extfns.h]
#ifndef _extfns_h
#define _extfns_h

sc_int<8> mult(sc_int<8> x, sc_int<8> y);

#endif
\end{lstlisting}

\begin{lstlisting}[language=C,title=File extfns.cpp]
#include <systemc.h>

sc_int<8> mult(sc_int<8> x, sc_int<8> y) 
{
  sc_int<8> res = x*y;
  return res;
}
\end{lstlisting}

\medskip
Things are similar for the \textbf{VHDL} backend. Here the generated file \texttt{scale\_act.vhd}
will refer to package named \texttt{work.extfns} which is supposed to contain the definition of the
\texttt{mult} function. Here's a possible definition for the corresponding file :

\begin{lstlisting}[language=vhdl,title=File extfns.vhd]
library ieee;
use ieee.std_logic_1164.all;
use ieee.numeric_std.all;

package extfns is
  function mult(a, b: signed(7 downto 0)) return signed;
end extfns;

package body extfns is
  function mult(a, b: signed(7 downto 0)) return signed is
    variable res : signed(15 downto 0);
  begin
     res := a * b;
     return res(7 downto 0);
  end;
end package body extfns;
\end{lstlisting}

\medskip
Programs making using of external functions have to provide a Caml implementation of the
corresponding functions in order to be simulated\footnote{In theory, the simulator could use the C++
implementation but this is currently not implemented.}. Integration of this code within the
simulator is done is done using the dynamic linking facilities offered by the Caml runtime\footnote{simply
because the simulator itself is written \ldots in Caml.}. For this a file \texttt{extfns.ml} must be
written (and accessible from the directory from which the simulator is run), ``registering'' the
corresponding functions. Here's the contents of this file for the \texttt{scale} example :

\begin{lstlisting}[language=caml,title=File extfns.ml]
let mult [x;y] = x * y
let _ = Foreign.register "mult" mult
\end{lstlisting}

The first argument to the function \texttt{Foreign.register} is the name by which the \texttt{mult}
function will be called in the \caph program (same name here, but another name could be used in case
of name conflicts).

Due to technical limitations, interfacing of foreign functions at the simulator level is limited to
functions taking and returning integer type(s).


%%% Local Variables: 
%%% mode: latex
%%% TeX-master: "caph"
%%% End: 
