%%%%%%%%%%%%%%%%%%%%%%%%%%%%%%%%%%%%%%%%%%%%%%%%%%%%%%%%%%%%%%%%%%%%%%%%%%%%%%%%%%%%%%%%%
%%                                                                                     %%
%%                This file is part of the CAPH Compiler distribution                  %%
%%                            http:%/caph.univ-bpclermont.fr                           %%
%%                                                                                     %%
%%                                  Jocelyn SEROT                                      %%
%%                         Jocelyn.Serot@univ-bpclermont.fr                            %%
%%                                                                                     %%
%%         Copyright 2011-2018 Jocelyn SEROT.  All rights reserved.                    %%
%%  This file is distributed under the terms of the GNU Library General Public License %%
%%      with the special exception on linking described in file ..%LICENSE.            %%
%%                                                                                     %%
%%%%%%%%%%%%%%%%%%%%%%%%%%%%%%%%%%%%%%%%%%%%%%%%%%%%%%%%%%%%%%%%%%%%%%%%%%%%%%%%%%%%%%%%%

\chapter{Typing}
\label{chap:typing}

\newcommand{\sy}[1]{\mathbf{#1}}
\newcommand{\sem}[1]{\mathsf{#1}}

\todo{type abbrevs}

This section gives the formal typing rules for the so-called \emph{Core} \caph language defined in
Chap.~\ref{chap:abssyn}. 

\medskip
The type language is fairly standard. A type $\tau$ is either\footnote{Functional, product and
  enumerated types are here treated specially although they could be handled as "normal" constructed
types.}
\begin{itemize}
\item a type variable $\alpha$
\item a constructed type $\chi\ \tau_1 \ldots \tau_n$, 
\item a functional type $\tau_1 \rightarrow \tau_2$,
\item a product type $\tau_1 \times \ldots \times \tau_n$,
\end{itemize}
\note{Should we write $\varepsilon=\{c_1,\ldots,c_n\}$, i.e. make enumerated types keep track of
  their constructors ?}
A type schema $\sigma$ is either
\begin{itemize}
\item a type $\tau$,
\item a type scheme $\forall \alpha. \sigma$.
\end{itemize}

\medskip
For simplicity, conversions from type schemes to types (instanciation)
and from types to type schemes (generalisation) have been left implicit since the rules are
completely standard.

\medskip
Typing occurs in the context of a \emph{typing environment} consisting of :
\begin{itemize}
\item a type environment T, recording type and value constructors (\emph{tycons} and \emph{ctors}
  resp.),
\item a variable environments V, mapping identifiers to (polymorphic) types.
\end{itemize}

\medskip
The initial type environment $TE_0$ records the type of the \emph{builtin} type and value
constructors :

$\begin{array}{lcl}
\TE_0.tycons & = & [ \sy{int}\mapsto \sem{Int}, \sy{bool}\mapsto \sem{Bool}, \\
%             &   &   \sy{array}\mapsto \alpha \rightarrow \sem{Array}~\alpha, \\
             &   &   \sy{dc}\mapsto \alpha \rightarrow \sem{Dc}~\alpha ] \\
\TE_0.ctors & = & [ \sy{0} \mapsto \sem{Int}, \sy{1} \mapsto \sem{Int}, \ldots, \\
            &   & \sy{true} \mapsto \sem{Bool}, \sy{false} \mapsto \sem{Bool} \\
            &   & \sy{SoS} \mapsto \sem{Dc}~\alpha, \sy{EoS} \mapsto \sem{Dc}~\alpha, \sy{Data} \mapsto \alpha \rightarrow \sem{Dc}~\alpha ] \\
\end{array}$

\note{Internally, there's no \texttt{Signed} nor \texttt{Unsigned} type constructors. These types
  are represented as annotated \texttt{Int}s. This avoids having to define distinct versions of the
  basic primitives like \texttt{+} or \texttt{-}.}

The initial variable environment $VE_0$ contains the types of the expression-level builtin
primitives.

$\begin{array}{lcl}
\VE_0 & = & [ \sy{+} : \sem{Int} \times \sem{Int} \rightarrow \sem{Int}, \\
      &   &   \sy{=} : \sem{Int} \times \sem{Int} \rightarrow \sem{Bool}, \ldots ] \\
\end{array}$

\section{Notations}
\label{sec:typing-notations}

Both type and variable \emph{environments} are viewed as partial maps (from identifiers to types and
from type constructors to types resp.). If $E$ is an environment, the domain and co-domain of $E$
are denoted by $\text{dom}(E)$ and $\text{codom}(E)$ respectively.  The empty environment will be
written $\emptyenv$.  $[x \mapsto y]$ denotes the singleton environment mapping $x$ to $y$.  We will
note $E(x)$ the result of applying the underlying map to $x$ (for ex. if $E$ is $[x \mapsto y]$ then
$E(x)=y$). We denote by $E[x \mapsto y]$ the environment that maps $x$ to $y$ and behaves like $E$
otherwise. $E \oplus E'$ denotes the environment obtained by adding the mappings of $E'$ to those of
$E$. If $E$ and $E'$ are not disjoints, then the mappings of $E$ are ``shadowed'' by those of
$E'$. We will also use a specialized version of the $\oplus$ merging
operator, denoted $\replace$, for which shadowing is replaced by type unification: if $x$ appears
both in $E$ and $E'$, and maps respectively to $\sigma$ and $\sigma'$, then it will map to
$\text{unify}_E(\sigma,\sigma')$ in $E \replace E'$
(this simplifies the description of the semantics of definitions for outputs).
$E \ominus E'$ denotes the environment obtained by removing any mapping $\lbrace x \mapsto y \rbrace$ for which $x
\in dom(E')$ from E.

When an environment $E$ is composed of several sub-environments $E'$,$E''$, \ldots, we note $E = \{
E', E'', \ldots \}$ and use the dot notation to access these sub-environments (ex: $E.E'$).

\medskip
For convenience and readability, we try to adhere to the following
naming conventions throughout this chapter :

\begin{center}
\begin{tabular}[c]{c|c}
  \textbf{Meta-variable} & \textbf{Meaning} \\ \hline
  $\TE$         & Type environment \\ 
  $\VE$         & Variable environment \\ 
  $ty$          & Type expression \\ 
  $\tau$        & Type \\ 
  $\sigma$      & Type scheme \\ 
  $\alpha$      & Type variable \\ 
   $\chi$           & Type constructor \\ 
   $c$           & Value constructor \\ 
   $\text{id}$    & Identifier \\ 
   $\emph{npat}$   & Pattern (network level) \\ 
   $\emph{nexp}$   & Network-level expression \\ 
   $\emph{qpat}$   & Qualified rule pattern (actor level) \\
   $\emph{rpat}$   & Rule pattern (actor level) \\
   $\emph{qexp}$   & Qualified rule expression (actor level) \\
   $\emph{exp}$   & Expression \\
   $\emph{qual}$   & Rule qualifier (actor level)
\end{tabular}
\end{center}

\medskip
Syntactical terminal symbols are written in $\mathbf{bold}$. Non terminals in $\emph{italic}$. Types
values are written in $\mathsf{serif}$.

\newpage
\section{Typing rules}
\label{sec:typing-rules}

\subsection{Programs}
\label{sec:typing-progr}


\ruleheader{\TE,\VE \vdash \text{Program} \gives \VE'}

\infrule[Program]
%{ \TE_0 \vdash \emtxt{tydecls} \gives \TE \\
{  \TE, \VE_0 \vdash \emtxt{valdecls} \gives \VE_v \\
  \andalso \TE, \VE_0 \oplus \VE_v \vdash \emtxt{actdecls} \gives \VE_a \\
  \andalso \TE \vdash \emtxt{iodecls} \gives \VE_i, \VE_o \\
  \andalso \TE, \VE_0 \oplus \VE_v \oplus \VE_a \oplus \VE_i \vdash \emtxt{netdecls} \gives \VE'}
%{\TE_0, \VE_0 \vdash \sy{program} ~\emtxt{tydecls} ~\emtxt{valdecls} ~\emtxt{actdecls}
{\TE_0, \VE_0 \vdash \sy{program} ~\emtxt{valdecls} ~\emtxt{actdecls}
  ~\emtxt{iodecls} ~\emtxt{netdefns} \gives \VE' \replace \VE_o}

% \subsection{Type declarations}
% \label{sec:typing-type-decls}

% \ruleheader{\TE~\vdash \text{TyDecls} \gives \TE'}

% \infrule[TyDecls]
% {\forallin{i}{1}{n},~~ \TE~\vdash \emtxt{tydecl}_i \gives \TE_i}
% {\TE, \VE~ \vdash \emtxt{tydecl}_1~\ldots~\emtxt{tydecl}_n \gives \C{i=1}{n}{\TE_i}}

% \infrule[TyDecl]
% { \TE~\vdash \emtxt{ty} \gives \tau }
% {\TE~ \vdash \mathtt{type}~\txt{id}~\mathtt{=}~\emtxt{ty} \gives [\txt{id} \mapsto \tau]}

\subsection{Value declarations}
\label{sec:typing-value-decls}

\ruleheader{\TE, \VE~\vdash \text{ValDecls} \gives \VE'}

\infrule[ValDecls]
{\forallin{i}{1}{n},~~ \TE,\VE_{i-1}~\vdash \emtxt{valdecl}_i \gives \VE_i,~~ \VE_0=\VE}
{\TE, \VE~ \vdash \emtxt{valdecl}_1~\ldots~\emtxt{valdecl}_n \gives {\VE_n}}

\infrule[ConstDecl1]
{ \TE, \VE~\vdash \emtxt{exp} \gives \tau }
{\TE, \VE~ \vdash \mathtt{const}~\txt{id}~\mathtt{=}~\emtxt{exp} \gives [\txt{id} \mapsto \tau]}

\infrule[ConstDecl2]
{ \TE, \VE~\vdash \emtxt{exp} \gives \tau \andalso \TE~\vdash \ty \gives \tau' \quad \emtxt{coercible}(\tau,\tau') }
{\TE, \VE~ \vdash \mathtt{const}~\txt{id}~\mathtt{=}~\emtxt{exp} ~:~ \ty \gives [\txt{id} \mapsto
  \tau']}

\infrule[FunDecl1]
{\forallin{i}{1}{n},~\vdash \emtxt{pat}_i, \tau_i \gives \VE_i \andalso \VE'=\C{i=1}{n}{\VE_i}\\
\TE,\VE \oplus \VE'~\vdash \emtxt{exp} \gives \tau'}
{\TE,\VE~ \vdash \mathtt{fun}~\txt{fid}~\tuplen{\emtxt{pat}}~\mathtt{\rightarrow}~\emtxt{exp} \gives
  [\txt{fid} \mapsto \tuplen{\tau} \rightarrow \tau']}

\infrule[FunDecl2]
{\forallin{i}{1}{n},~\vdash \emtxt{pat}_i, \tau_i \gives \VE_i \andalso \VE'=\C{i=1}{n}{\VE_i}\\
\TE,\VE \oplus \VE'~\vdash \emtxt{exp} \gives \tau' \\
 \TE~\vdash \ty \gives \tau'' \quad \emtxt{coercible}(\tuplen{\tau} \rightarrow \tau',\tau'') }
{\TE,\VE~ \vdash \mathtt{fun}~\txt{fid}~\tuplen{\emtxt{pat}}~\mathtt{\rightarrow}~\emtxt{exp} ~:~ \ty  \gives
  [\txt{fid} \mapsto \tau'']}

% \infrule[ExtFunDecl]
% { \TE~\vdash \ty \gives \tau \andalso \TE~\vdash \ty' \gives \tau'}
% {\TE,\VE~ \vdash \mathtt{extern}~\txt{id}~\mathtt{:}~\ty~\mathtt{\rightarrow}~\ty' \gives [\txt{id}
%   \mapsto \tau \rightarrow \tau']}

\newpage
where 

\infax{\vdash \emtxt{pat}, \tau \gives \VE}

means that declaring \emph{pat} with type $\tau$ creates the variable environment $\VE$.

\medskip
The predicate \emph{coercible} tells whether the infered type $\tau$ can be cast to the declared type
$\tau'$. The coercibility relation has been defined in Sec.~\ref{sec:e-coerce}.

% \subsubsection{Function patterns}
% \label{sec:typing-fun-patterns}

% The following rules are used to handle pattern binding in function declarations

% \ruleheader{\vdash_f \text{FPat},\tau \gives \VE}

% \infrule[FPatId]
% {}
% {~\vdash_f \txt{id},~ \tau \gives [\txt{id} \mapsto \tau]}

% \infrule[FPatTuple]
% {\forallin{i}{1}{n}, ~~\vdash_f \emtxt{fpat}~_i, \tau_i \gives \VE_i}
% {\vdash_f \emtxt{fpat}~_1, \ldots \emtxt{fpat}~_n,~~ \tau_1 \times \ldots \times \tau_n \gives \C{i=1}{n}{\VE_i}}
% where 

% \infax{\vdash_f \emtxt{pat}, \tau \gives \VE}

% means that declaring \emph{pat} with type $\tau$ creates the variable environment $\VE$.

\subsection{Actor declarations}
\label{sec:typing-actors}

\ruleheader{\TE, \VE ~\vdash \text{ActorDecls} \gives \VE'}

\infrule[ActorDecls]
{\forallin{i}{1}{n},~~ \TE,\VE~\vdash \emtxt{actdecl}_i \gives \VE_i}
{\TE,\VE~ \vdash \emtxt{actdecl}_1~\ldots~\emtxt{actdecl}_n \gives \C{i=1}{n}{\VE_i}}

\infrule[ActorDecl1]
{          \emtxt{params} \not = \emptyset \\
           \TE \vdash \emtxt{params} \gives \tau,\VE_p \\
  \andalso \TE \vdash \emtxt{ins} \gives \tau',\VE_i \\
  \andalso \TE \vdash \emtxt{outs} \gives \tau'',\VE_o \\
  \andalso \TE,\VE \oplus \VE_p \vdash \emtxt{localvars} \gives \VE_v,\TE_v \\
  \andalso \TE \oplus \TE_v, \VE \oplus \VE_p \oplus \VE_v \oplus \VE_i \oplus \VE_o \vdash \emtxt{rules} \gives \VE_r}
{\TE,\VE~ \vdash
  \actorr~\txt{id}~\emtxt{params}~\emtxt{ins}~\emtxt{outs}~\emtxt{localvars}~\emtxt{rules}
  \gives [\txt{id} \mapsto \tau \rightarrow \tau' \rightarrow \tau'']}

\infrule[ActorDecl2]
{          \emtxt{params} = \emptyset \\
  \andalso \TE \vdash \emtxt{ins} \gives \tau',\VE_i \\
  \andalso \TE \vdash \emtxt{outs} \gives \tau'',\VE_o \\
  \andalso \TE,\VE \oplus \VE_p \vdash \emtxt{localvars} \gives \VE_v,\TE_v \\
  \andalso \TE \oplus \TE_v, \VE \oplus \VE_p \oplus \VE_v \oplus \VE_i \oplus \VE_o \vdash \emtxt{rules} \gives \VE_r}
{\TE,\VE~ \vdash
  \actorr~\txt{id}~\emtxt{params}~\emtxt{ins}~\emtxt{outs}~\emtxt{localvars}~\emtxt{rules}
  \gives [\txt{id} \mapsto \tau' \rightarrow \tau'']}

Note that the type assigned to an actor only depends on its interface (parameters, inputs and outputs).
An actor \texttt{a} declared as

\medskip
$\mathtt{actor}~a~(p_1:t_1,\ldots,p_k:t_k)~\mathtt{in}~(i_1:t'_1,\ldots,i_m:t'_m)~\mathtt{out}~(o_1:t''_1,\ldots,o_n:t''_n)$

\medskip
will be assigned type

\medskip
$t_1\times\ldots\times t_k \rightarrow t'_1\times\ldots\times t'_m \rightarrow t''_1\times\ldots\times t''_n$

\medskip
whereas an actor \texttt{a} declared as

\medskip
$\mathtt{actor}~a~\mathtt{in}~(i_1:t'_1,\ldots,i_m:t'_m)~\mathtt{out}~(o_1:t''_1,\ldots,o_n:t''_n)$

\medskip
will be assigned type

\medskip
$t'_1\times\ldots\times t'_m \rightarrow t''_1\times\ldots\times t''_n$

\medskip
% The rule schema may refer to identifiers bound to parameters, inputs, outputs or local variables.
% When typing rules, only the parameters are visible. Inputs, outputs and local variables are accessed
% indirectly by pattern matching against the rule schema.

\ruleheader{\TE~\vdash \text{ActParams} \gives \tau,\VE}

\infrule[ActParams]
{\forallin{i}{1}{n},~~ \TE~\vdash \emtxt{param}_i \gives \tau_i,\VE_i}
{\TE~ \vdash \emtxt{param}_1~\ldots~\emtxt{param}_n \gives \tau_1\times\ldots\times\tau_n,~ \C{i=1}{n}{\VE_i}}

\infrule[ActParam]
{\TE~\vdash \ty \gives \tau}
{\TE~\vdash \txt{id} ~:~ \ty \gives \tau,~ [\txt{id} \mapsto \tau]}

% The predicate \emph{valid\_param\_type} is used to accept only parameters with type 
% \begin{itemize}
% \item $\tau$, where $\tau$ is a \emph{scalar type} ($\mathsf{Int}$ or $\mathsf{Bool}$),
% \item $\mathsf{Array}~\tau$ where $\tau$ is a scalar type,
% \item $\tau_1 \times \ldots \times \tau_n \rightarrow \tau_{n+1}$, where $\tau$ is a scalar type.
% \end{itemize}

\medskip\ruleheader{\TE~\vdash \text{ActIOs} \gives \tau,\VE'}

\infrule[ActIOs]
{\forallin{i}{1}{n},~~ \TE~\vdash \emtxt{io}_i \gives \tau_i, \VE_i}
{\TE~ \vdash \emtxt{io}_1~\ldots~\emtxt{io}_n \gives \tau_1 \times \ldots \times \tau_n,~ \C{i=1}{n}{\VE_i}}

\infrule[ActIO]
{\TE~\vdash \ty \gives \tau}
{\TE~\vdash \txt{id} ~:~ \ty \gives \tau,~ [\txt{id} \mapsto \tau]}

% The predicate \emph{valid\_io\_type} is used to accept only parameters with type $\tau$ or
% $\mathsf{Dc}~\tau$ where $\tau$ is a scalar type.

\ruleheader{\VE,\TE~\vdash \text{ActVars} \gives \VE',\TE'}

\infrule[ActVars]
{\forallin{i}{1}{n},~~ \VE,\TE~\vdash \emtxt{var}_i \gives \VE_i, \TE_i}
{\VE,\TE~ \vdash \emtxt{var}_1~\ldots~\emtxt{var}_n \gives \C{i=1}{n}{\VE_i}, \C{i=1}{n}{\TE_i}}

\infrule[ActVar]
{\TE~\vdash \ty \gives \tau,\TE'}
{\VE,\TE~\vdash \txt{id} ~:~ \ty \gives \tau, [\txt{id} \mapsto \tau], \TE'}

\infrule[ActVar']
{\TE~\vdash \ty \gives \tau,\TE' \quad \TE,\VE~\vdash \emtxt{exp} \gives \tau' \quad
  \emtxt{coercible}(\tau',\tau)}
{\VE,\TE~\vdash \txt{id} ~:~ \ty ~=~ \emtxt{exp} \gives [\txt{id} \mapsto \tau], \TE'}

% The predicate \emph{valid\_var\_type} is used to accept only parameters with type 
% \begin{itemize}
% \item $\tau$, where $\tau$ is a \emph{scalar type} ($\mathsf{Int}$ or $\mathsf{Bool}$) or an enumerated type,
% \item $\mathsf{Array}~\tau$ where $\tau$ is a scalar type.
% \end{itemize}

\medskip
Typing of
actor local variables is the only situation where the type environment TE can be augmented. This
happens when a variable is declared with an enumerated type. In this case, the (nullary) enumerated
constants are added, as nullary value constructors, to the type environment. Such declarations can be viewed as
purely local type declarations, since the scope of the declared constructors is limited to the rule set of the enclosing actor).

% \medskip
% A special set of rules deals with the values which can be used to initialize local
% variables\footnote{Enumerated constants are only valid for local variables.}

% \ruleheader{\TE~\vdash \text{Const} \gives \tau}

% \infrule[ScalarConst]
% {}
% {\TE,\VE \vdash \txt{int}/\txt{bool} \gives \mathsf{Int}/\mathsf{Bool}}

% \infrule[ArrayConst]
% {\forallin{i}{1}{n},~ \TE~\vdash \emtxt{cst}_i \gives \tau \quad \emtxt{valid\_scalar\_type}~ \tau}
% {\TE~ \vdash \mathbf{[}~\emtxt{cst}_1~\ldots~\emtxt{cst}_n\mathbf{]} \gives \mathsf{Array}~\tau}

% \infrule[EnumConst]
% {\TE.ctors(c)=\tau}
% {\TE~ \vdash \txt{c} \gives \tau}


\subsubsection{Actor rules}
\label{sec:typing-actor-rules}

\ruleheader{\TE,\VE~\vdash \text{ActRules} \gives \VE'}

\infrule[ActRules]
{\forallin{i}{1}{n},~~ \TE,\VE~\vdash \emtxt{rule}_i \gives \tau_i}
{\TE,\VE~ \vdash \emtxt{rule}_1~\ldots~\emtxt{rule}_n \gives [ 1 \mapsto \tau_1, \ldots, n \mapsto \tau_n]}

Typing the rule set produces a environment mapping rule numbers to types. Each rule can have a
distinct type. 

\ruleheader{\TE,\VE~\vdash \text{ActRule} \gives \tau \rightarrow \tau'}

\infrule[ActRule]
{ \forallin{i}{1}{m},~~\TE,\VE~\vdash \emtxt{qpat}_i \gives \tau_i, \VE'_i \\
  \forallin{j}{1}{n},~~\TE,\VE\oplus\C{i=1}{n}{\VE'_i} \vdash \emtxt{qexp}_j \gives \tau'_j}
{\TE,\VE~ \vdash \langle \emtxt{qpat}_1,\ldots,\emtxt{qpat}_m\rangle
  ~\rightarrow~\langle\emtxt{qexp}_1,\ldots,\emtxt{qexp}_n\rangle \gives
  \tau_1\times\ldots\times\tau_m \rightarrow \tau'_1\times\ldots\times\tau'_n}

\subsubsection{Rule patterns}
\label{sec:typing-rule-patterns}

Typing a (qualified) rule pattern gives a type and an environment, mapping each identifier
introduced in the pattern to a type. The type is retrieved using the pattern qualifier.

\ruleheader{\TE,\VE~\vdash \text{QPat} \gives \tau,\VE}

\infrule[RPatConstInt]
{\VE(\txt{id})=\mathsf{Int}}
{\TE,\VE~\vdash \ttuple{\txt{id}}{\txt{int}} \gives \mathsf{Int},~  \emptyenv}

\infrule[RPatConstBool]
{\VE(\txt{id})=\mathsf{Bool}}
{\TE,\VE~\vdash \ttuple{\txt{id}}{\txt{int}} \gives \mathsf{Bool},~  \emptyenv}

\infrule[RPatVar]
{\VE(\txt{id})=\tau}
{\TE,\VE~\vdash \ttuple{\txt{id}}{\txt{var}} \gives \tau,~ [\txt{var} \mapsto \tau]}

\infrule[RPatWild]
{\VE(\txt{id})=\tau}
{\TE,\VE~\vdash \ttuple{\txt{id}}{\txt{\_}} \gives \tau,~ \emptyenv}

\infrule[RPatCon0]
{\VE(\txt{id})=\tau \andalso \TE.ctors(c)=\tau}
{\TE,\VE~\vdash \ttuple{\txt{id}}{\txt{c}~\emptytuple} \gives \tau,~ \emptyenv}

\infrule[RPatCon]
{\VE(\txt{id})=\tau \andalso \TE.ctors(c) = \seqopn{\tau}{\times} \rightarrow \tau \\
 \forallin{i}{1}{n}, \TE,\VE~\vdash \emtxt{qpat}_i \gives \tau_i,~ \VE_i}
{\TE,\VE~\vdash \ttuple{\txt{id}}{\txt{c}~\tuplen{\emtxt{qpat}}} \gives \tau,~ \C{i=1}{n}{\VE_i}}

\subsection{Expressions}
\label{sec:typing-expressions}

\ruleheader{\TE,\VE~ \vdash \text{QExp} \gives \tau}
\infrule[]
{\TE,\VE \vdash \emtxt{exp} \gives \tau \andalso \VE(\txt{id}) = \tau}
{\TE,\VE \vdash \ttuple{\txt{id}}{\emtxt{exp}} \gives \tau}

The infered type for a qualified expression must match the type assigned to the corresponding output
or local variable.

\ruleheader{\TE,\VE~ \vdash \text{Exp} \gives \tau}

\infrule[EConst]
{}
{\TE,\VE \vdash \txt{int}/\txt{bool} \gives \mathsf{Int}/\mathsf{Bool}}

\infrule[EVar]
{\VE(id) = \tau}
{\TE,\VE \vdash \txt{id} \gives \tau}

\infrule[EWild]
{}
{\TE,\VE \vdash \txt{\_} \gives \alpha}

\infrule[ECon0]
{\TE.ctors(c)=\tau}
{\TE,\VE ~\vdash \txt{c}~\emptytuple \gives \alpha}

\infrule[ECon]
{\TE.ctors(c) = \seqopn{\tau}{\times} \rightarrow \tau \\
 \forallin{i}{1}{n},~~ \TE, \VE \vdash \emtxt{exp}_i \gives \tau_i}
{\TE,\VE ~\vdash \txt{c}~\tuplen{\emtxt{exp}} \gives \tau}

\note{If $c=\sem{Data}$, $\tau_1$ must be a scalar type !}

% \infrule[REEnum]
% {\TE.ctors(c)=\tau}
% {\TE,\VE \vdash \txt{c} \gives \tau}

% \infrule[RESEOS]
% {}
% {\TE,\VE \vdash \mathbf{SoS}/\mathbf{EoS} \gives \mathsf{Dc}~\alpha}

% \infrule[REData]
% {\TE,\VE \vdash exp \gives \tau \quad \emtxt{valid\_scalar\_type}~\tau}
% {\TE,\VE \vdash \mathbf{data}~\emtxt{exp} \gives \mathsf{Dc}~\alpha}

\infrule[EApp]
{\VE(id)=\tau_1 \times \ldots \times \tau_n \rightarrow \tau' \\
 \forallin{i}{1}{n}, ~~\TE,\VE \vdash ~\emtxt{exp}_i \gives  \tau_i}
{\TE,\VE \vdash \text{id}\lparenn \emtxt{exp}_1,~\ldots,~\emtxt{exp}_n\rparenn \gives \tau'}

\infrule[ECond]
{\TE,\VE \vdash \emtxt{exp} \gives \mathsf{Bool} \andalso \TE,\VE \vdash \emtxt{exp}_1 \gives \tau
  \andalso \VE,\TE \vdash \emtxt{exp}_2 \gives \tau}
{\TE,\VE \vdash \mathbf{if}~ \emtxt{exp}~ \mathbf{then} ~\emtxt{exp}_1 ~\elsee ~\emtxt{exp}_2  \gives \tau}

\infrule[ELet]
{\vdash_p \text{id},\tau' \gives \VE' \andalso \TE,\VE \vdash \emtxt{exp}_2 \gives \tau' \andalso
  \TE,\VE \oplus \VE'' \vdash \emtxt{exp}_1 \gives \tau}
{\TE,\VE \vdash \mathbf{let}~\text{id}~\mathbf{=}~\emtxt{exp}_2~\mathbf{in}~\emtxt{exp}_1 \gives  \tau}

% \infrule[AEFun]
% {\vdash_p \text{id},\tau \gives \VE' \andalso \TE,\VE \oplus \VE' \vdash \emtxt{exp} \gives \tau'}
% {\TE,\VE \vdash \mathbf{function}~\text{id}~\mathbf{\rightarrow}~\emtxt{exp} \gives \tau \rightarrow \tau'}

% \infrule[EArrRd]
% {\VE(\txt{id}) = \mathsf{Array}~\tau \andalso \TE,\VE \vdash \emtxt{exp} \gives \mathsf{Int}}
% {\TE,\VE \vdash \txt{id} \mathbf{[} \emtxt{exp} \mathbf{]} \gives \tau}

% \infrule[EArrUpd]
% {\VE(\txt{id}) = \mathsf{Array}~\tau \andalso \TE,\VE \vdash \emtxt{exp}_1 \gives \mathsf{Int} \andalso
%   \TE,\VE \vdash \emtxt{exp}_2 \gives \tau}
% {\TE,\VE \vdash \txt{id} \mathbf{[} \emtxt{exp}_1 \leftarrow \emtxt{exp}_2 \mathbf{]} \gives
%   \mathsf{Array}~\tau}

\infrule[ECoerce]
{\TE,\VE \vdash ~\emtxt{exp} \gives \tau \andalso \TE \vdash \emtxt{ty} \gives \tau' \andalso
  \emtxt{coercible}(\tau,\tau') }
{\TE,\VE ~\vdash \emtxt{exp}~\mathbf{:}~\emtxt{ty} \gives \tau'}

% The predicate \emph{coercible} tells whether the infered type $\tau$ can be cast to the declared type
% $\tau'$. The coercibility relation has been defined in Sec.~\ref{sec:e-coerce}.

% \begin{figure}[h]
%   \centering
% \begin{tabular}[c]{|c|c|c|c|c|c|c|}
%                      & \texttt{signed<n>} & \texttt{unsigned<n>} & \texttt{int} & \texttt{bool} & \\ \hline
% \texttt{signed}      & Y                  & Y                    & Y            &               & \\ \hline
% \texttt{unsigned}    &                    & Y                    & Y            &               & \\ \hline
% \texttt{int}         & Y                  & Y                    & Y            &               & \\ \hline
% \texttt{bool}        &                    &                      &              & Y             & \\ \hline
% \end{tabular}
%   \caption{Coercibility relation}
%   \label{fig:coercible}
% \end{figure}

\subsection{IOs}
\label{sec:typing-io-declarations}

\ruleheader{\TE~\vdash \text{IoDecls} \gives \VE_i,\VE_o}

\infrule[IoDecls]
{\forallin{i}{1}{n},~~ \TE~\vdash \emtxt{iodecl}_i \gives \VE_i,\VE_o}
{\TE~ \vdash \emtxt{iodecl}_1~\ldots~\emtxt{iodecl}_n \gives \C{i=1}{n}{\VE_i},\C{i=1}{n}{\VE_o}}

\infrule[StreamDecl]
{\TE~\vdash \ty \gives \tau}
{\TE~ \vdash \streamm~\txt{id}~\ty~\txt{from}~\txt{id'} \gives [\txt{id} \mapsto \tau], \emptyenv}

\infrule[StreamDecl]
{\TE~\vdash \ty \gives \tau}
{\TE~ \vdash \streamm~\txt{id}~\ty~\txt{to}~\txt{id'} \gives \emptyenv, [\txt{id} \mapsto \tau]}

\subsection{Network declarations}
\label{sec:typing-network-declarations}

\ruleheader{\TE, \VE~ \vdash \text{NetDecls} \gives \VE'}

\infrule[NetDecls]
{\VE_0=\VE \andalso \forallin{i}{1}{n},~~ \TE, \VE_{i-1}~\vdash \emtxt{netdecl}_i \gives \VE_i}
{\text{\VE}~ \vdash \emtxt{netdecl}_1~\ldots~\emtxt{netdecl}_n \gives \VE_n}

\ruleheader{\TE, \VE~ \vdash \text{NetDecl} \gives \VE'}

\infrule[NetDecl]
{\forallin{i}{1}{n},~~ \TE,\VE \vdash \emtxt{nbind} \gives \VE_i \andalso \VE'=\C{i=1}{n}{\VE_i}}
{\TE, \VE \vdash \mathbf{net}~ \tuplen{nbind} \gives \VE \oplus \VE'}

\infrule[NetRecDecl]
{\forallin{i}{1}{n},~~ \TE,\VE \oplus \VE' \vdash \emtxt{nbind} \gives \VE_i \andalso \VE'=\C{i=1}{n}{\VE_i}}
{\TE, \VE \vdash \mathbf{net}~ \mathbf{rec}~ \tuplen{nbind} \gives \VE \oplus \VE'}

\ruleheader{\TE, \VE~ \vdash \text{NetBind} \gives \VE'}

\infrule[NetBind]
{\TE, \VE \vdash \emtxt{nexp} \gives \tau \andalso \vdash_p \emtxt{npat},\tau \gives \VE'}
{\TE, \VE \vdash \emtxt{npat}~ \mathbf{=}~ \emtxt{nexp}~ \gives \VE'}

% \infrule[RecNDecl]
% {\VE \oplus \VE' \vdash \emtxt{nexp}~ \gives \tau  \andalso \vdash_p \emtxt{npat}~,\tau \gives \VE'}
% {\TE, \VE \vdash \mathbf{ndef}~ \mathbf{rec}~ \emtxt{npat}~ \mathbf{=}~ \emtxt{nexp}~ \gives \VE \oplus \VE'}

\note{Can we omit \TE from here ?}

\subsubsection{Patterns}
\label{sec:typing-patterns}

The following rules are used to handle pattern binding at the network level :

\ruleheader{\vdash_p \text{NPat},\tau \gives \VE}

\infrule[NPatVar]
{}
{\vdash_p \txt{id}, \tau \gives [\txt{id} \mapsto \tau]}

\infrule[NPat Tuple]
{\forallin{i}{1}{n}, ~~\vdash_p \emtxt{npat}~_i, \tau_i \gives \VE_i}
{\vdash_p \leftp \emtxt{npat}~_1, \ldots \emtxt{npat}~_n \rightp, \tau_1 \times \ldots \times \tau_n \gives \C{i=1}{n}{\VE_i}}

% \infrule[NPat Cons]
% {\vdash_p \emtxt{npat}_1, \tau \gives \VE \andalso \vdash_p \emtxt{npat}_2,\mathsf{List}~{\tau} \gives \VE'}
% {\vdash_p \mathbf{cons}(\emtxt{npat}_1,\emtxt{npat}_2), \mathsf{List}~{\tau} \gives \VE \oplus \VE'}

% \infrule[NPat Nil]
% {}
% {\vdash_p \mathbf{nil}, \mathsf{List}~{\alpha} \gives \emptyenv}

where 

\infax{\vdash_p \emtxt{npat}, \tau \gives \VE}

means that declaring \emph{npat} with type $\tau$ creates the variable environment $\VE$.

\subsubsection{Network Expressions}
\label{sec:typing-network-expressions}

\ruleheader{\TE, \VE~ \vdash \text{NExp} \gives \tau}

% \infrule[NVar1]
% {\VE(id) = \tau}
% {\TE,\VE \vdash \txt{id} \langle \rangle \gives \tau}

% \infrule[NVar2]
% {\VE(id) = \tau_1 \times \ldots \times \tau_n \rightarrow \tau' ~~ \forallin{i}{1}{n}, ~~\TE,\VE \vdash ~\emtxt{exp}_i \gives \tau_i}
% {\TE,\VE \vdash \txt{id}~ \langle \emtxt{exp}_1 , ~\ldots ~\emtxt{exp}_n \rangle \gives \tau'}

% Rule \textsc{NVar1} applies when the identifier has been defined the network level or designates an actor
% with no parameter. Rule \textsc{NVar2} applies when the identifier refers to an actor with
% parameters. 

\infrule[NVar]
{\VE(id) = \tau}
{\TE,\VE \vdash \txt{id} \gives \tau}

\infrule[NConst]
{}
{\TE,\VE \vdash \txt{int}/\txt{bool} \gives \mathsf{Int}/\mathsf{Bool}}

\infrule[NTuple]
{\forallin{i}{1}{n}, ~~\VE \vdash ~\emtxt{nexp}_i \gives  \tau_i}
{\TE,\VE \vdash \mathbf{(} \emtxt{nexp}_1 \mathbf{,} ~\ldots ~\emtxt{nexp}_n \mathbf{)} \gives  \tau_1 \times \ldots \times \tau_n}

\infrule[NApp]
{\VE \vdash \emtxt{nexp}_1 \gives  \tau \rightarrow \tau' \andalso \VE \vdash \emtxt{nexp}_2 \gives \tau}
{\TE,\VE \vdash \emtxt{nexp}_1 ~\emtxt{nexp}_2 \gives  \tau'}

\infrule[NFun]
{\vdash_p \emtxt{npat}~, \tau \gives \VE' \andalso \VE \oplus \VE' \vdash \emtxt{nexp} \gives  \tau'}
{\TE,\VE \vdash \mathbf{function}~ \emtxt{npat}~ \rightarrow \emtxt{nexp} \gives  \tau \rightarrow \tau'}

\infrule[NLet]
{\vdash_p \emtxt{npat}~, \tau' \gives \VE' \andalso \VE \vdash \emtxt{nexp}_2 \gives  \tau' \andalso \VE \oplus \VE' \vdash \emtxt{nexp}_1 \gives  \tau}
{\TE,\VE \vdash \mathbf{let}~ \mathbf{nonrec}~ \emtxt{npat}~ \bftxt{=} \emtxt{nexp}_2~ \mathbf{in}~ \emtxt{nexp}_1 \gives  \tau}

\infrule[NLet Rec]
{\vdash_p \emtxt{npat}~, \tau' \gives \VE' \andalso \VE \oplus \VE' \vdash \emtxt{nexp}_2 \gives  \tau'
  \andalso \VE \oplus \VE' \vdash \emtxt{nexp}_1 \gives  \tau}
{\TE,\VE \vdash \mathbf{let}~ \mathbf{rec}~ \emtxt{npat}~ \bftxt{=} \emtxt{nexp}_2~ \mathbf{in}~ \emtxt{nexp}_1 \gives  \tau}

% \infrule[NCond]
% {\VE \vdash \emtxt{nexp} \gives \text{bool} \andalso \VE \vdash \emtxt{nexp}_1 \gives \tau \andalso \VE \vdash \emtxt{nexp}_2 \gives \tau}
% {\VE \vdash \mathbf{if}~ \emtxt{nexp}~ \mathbf{then}~ ~\emtxt{nexp}_1 ~\mathbf{then} ~\emtxt{nexp}_2  \gives  \tau}

% \infrule[NMatch]
% {\VE \vdash \emtxt{nexp}\gives\tau_1 \andalso \forallin{i}{1}{n},~~\vdash_p \emtxt{pat}~_i,\tau_1 \gives \VE_i
%   \andalso \VE
%   \oplus \VE_i \vdash \emtxt{nexp}_i \gives \tau_2}
% {\VE \vdash \mathbf{match}~ \emtxt{nexp} ~\mathbf{with}~ (\emtxt{pat}~_1 \rightarrow \emtxt{nexp}_1) \ldots (\emtxt{pat}~_n \rightarrow \emtxt{nexp}_n) \gives \tau_2}

% \infrule[NCons]
% {\VE \vdash \emtxt{nexp}~_1 \gives \tau \andalso \VE \vdash \emtxt{nexp}~_2 \gives \mathsf{List}~\tau}
% {\VE \vdash \mathbf{cons}(\emtxt{nexp}_1,\emtxt{nexp}_2) \gives  \mathsf{List}~\tau}

% \infrule[NNil]
% {}
% {\VE \vdash \mathbf{()} \gives \mathsf{List}~\alpha}

\subsection{Type expressions}
\label{sec:typing-type-expressions}

\ruleheader{\TE~\vdash \emtxt{ty} \gives \tau,\TE'}

\infrule[TyCon0]
{\TE.tycons(\chi)=\tau}
{\TE \vdash \chi \gives \tau, \emptyenv}

\infrule[TyCon]
{\TE.tycons(\chi)=\seqopn{\tau}{\times}\rightarrow\tau \\
 \forallin{i}{1}{n},~~ \TE \vdash \emtxt{ty}_i \gives \tau_i, \emptyenv}
{\TE \vdash \chi~\tuplen{\emtxt{ty}} \gives \tau, \emptyenv}

\infrule[TyTuple]
{\forallin{i}{1}{n}, ~~\TE \vdash ~\emtxt{ty}_i \gives  \tau_i, \emptyenv}
{\TE \vdash \emtxt{ty}_1 \times \ldots \times \emtxt{ty}_n \gives  \tau_1 \times \ldots \times \tau_n, \emptyenv}

\infrule[TyFun]
{\TE \vdash \emtxt{ty} \gives \tau, \emptyenv \andalso \vdash \emtxt{ty}' \gives \tau', \emptyenv}
{\TE \vdash \emtxt{ty}~ \rightarrow ~\emtxt{ty}' \gives  \tau \rightarrow \tau', \emptyenv}

\infrule[TyEnum]
{\text{tyname},\tau = \text{new\_tyname}()}
{\TE \vdash \mathbf{enum}\langle \txt{c}_1, \ldots, \txt{c}_n \rangle \gives \tau,
  \{tycons=[\text{tyname}
  \mapsto \tau]; ctors=[c_1 \mapsto \tau, \ldots, c_n \mapsto \tau]\}}

The function \emph{new\_tyname} generates a fresh type name and type constructor.

\ruleheader{\TE~\vdash \emtxt{ty} \gives \tau}

\infrule[]
{\TE \vdash \emtxt{ty} \gives \tau, \TE'}
{\TE \vdash \emtxt{ty} \gives \tau}

\medskip
The main originality here is the introduction of local, anonymous declarations for enumerated types.
These declarations are given when declaring local variables within actors. The scope of the
implicitely declared type is limited to the enclosing actor. As a result, typing a type expression
may result in an updated type environment, which is reflected in the signature of the above rules.

%%% Local Variables: 
%%% mode: latex
%%% TeX-master: "caph"
%%% End: 
